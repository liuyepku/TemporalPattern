
%%%
%%% Util Source Documentation
%%% Frank Hoeppner, DLR, Braunschweig, 1997-99
%%%

\documentclass{book}

%% source listing specific
\newif\ifsourcecode
%\sourcecodetrue
\sourcecodefalse
\newif\ifhackerdocs
%\hackerdocstrue
\hackerdocsfalse
\newif\ifdeveloper
%\developertrue
\developerfalse

%\usepackage{citesort}
%\usepackage{refman}
\usepackage{subfigure}
\usepackage{a4dutch}
%\descriptionleftfalse
%\descriptioncolonfalse
\input{tutorial}

\parskip=0.4\baselineskip \advance\parskip by 0pt plus 2pt
\parindent=0pt

\begin{document}
\bibliographystyle{abbrv}
%\sloppy

%%% 
%%% Title
%%%

\thispagestyle{empty}
\pagenumbering{roman}

\title{Basic Utility Functions\\[-3mm]
\rule{12cm}{1mm}\\ 
{\normalsize\sc \ifhackerdocs developer's \else user's \fi manual}}
\author{Frank H\"oppner}

\maketitle

%%%
%%% Dedication
%%%

\newpage This document is part of the {\tt utility} package. Maintainer of
this package is Frank H\"oppner. The C++-sources and this documentation have
been written using the {\sl literate programming} tool noweb
\cite{Ramsey:TS:94, Ramsey:WWW:noweb}. Please mail bug reports, suggestions,
comments and any other feedback to alias \url{frank.hoeppner@ieee.org}. The
package has been developed under Linux with the egcs-2.91.66
compiler.\index{compiler}

You can obtain this package from \url{http://www.fuzzy-clustering.de}.

{\tt utility} package Copyright \copyright\ 1997-2000 Frank H\"oppner

The programs in this package are free software; you can redistribute and/or
modify them under the terms of the GNU General Public License as published by
the Free Software Foundation; either version 2 of the License, or (at your
option) any later version.

These programs are distributed in the hope that they will be useful,
but WITHOUT ANY WARRANTY; without even the implied warranty of
MERCHANTABILITY or FITNESS FOR A PARTICULAR PURPOSE. See the GNU
General Public License for more details.

You should have received a copy of the GNU General Public License along with
this program; if not, write to the Free Software Foundation, Inc., 59 Temple
Place, Suite 330, Boston, MA 02111-1307 USA

%%%
%%% Contents
%%%

\tableofcontents
\newpage

%%%%
%%%% Hauptteil
%%%%

%%%
%%% Introduction
%%%

\cleardoublepage
%\pagestyle{noweb}
\pagenumbering{arabic}
\setcounter{page}{1}

\chapter*{Introduction}
\addcontentsline{toc}{chapter}{Introduction}

The {\tt utility} package covers frequently used constant definitions, simple
time measurement, stream analysis (chapter \ref{util:cha}), and several
functions that are useful in combination with the STL (standard template
library \cite{Josuttis:AW:1996}, chapter \ref{stl:cha}).

%\input{litprog}
%\input{compiler}

\input{utiltest}

%%%
%%% Utilities
%%%

\chapter{Utility Functions}
\label{util:cha}

This chapter treats constant definitions, floating point time functions, a
very simple character class, and functions for input stream analysis and
string conversion.

\input{constants}
\include{time}
\include{chars}
\include{utilstream}
\include{strconvert}

%%%
%%% STL Utilities
%%%

\chapter{STL Related Utilities}
\label{stl:cha}

The source code provided in this chapter deals with STL support. Some compiler
have their own \verb+bool+ type definition, others not. The STL comes along
with its own \verb+bool+ type definition. The file in section \ref{defbool:sec}
tries to find out whether you still have do define your bool or not. If you use
STL data types like \verb+map+ or \verb+set+, you can find some helpful
less-functors in section \ref{stlutil:sec}. Finally, we provide (default) i/o
functions for all STL data types in section \ref{stlstream:sec}.

\input{defbool}
\include{stlutil}
\include{stlstream}

%%%
%%% References
%%%

\newpage
\bibliography{abr,abc,def,ghi,jkl,mno,pqr,stu,vwxyz}
%prg,sourcedoc,dlr-ib}
\addcontentsline{toc}{chapter}{References}

%%%
%%% Index
%%%

\newpage
\addcontentsline{toc}{chapter}{Index}
\printindex
  
\end{document}

