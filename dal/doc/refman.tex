
\documentclass[a4paper]{book}
\usepackage[latin1]{inputenc}
\usepackage{german}
\usepackage{times}
\usepackage{a4wide}
\usepackage{makeidx}
\usepackage{fancyhdr}
\usepackage{graphicx}
\usepackage{float}
\usepackage{alltt}
\usepackage{doxygen}

\makeindex
\setcounter{tocdepth}{1}
\setlength{\footrulewidth}{0.4pt}
\begin{document}

\rfoot[\today]{DAL Referenz}
\lfoot[DAL Referenz]{\today}
\cfoot{}

%%% title
\thispagestyle{empty}
\title{DAL -- Referenzhandbuch\\[-3mm]
\rule{12cm}{1mm}\\[1cm]}
\author{Frank H\"oppner}
\maketitle

%%% body
\clearemptydoublepage
\pagenumbering{roman}
\tableofcontents
\clearemptydoublepage
\pagenumbering{arabic}

\chapter{Einf�hrung}

When writing your own programs, you should stick to the following
command order in your main function:

\begin{enumerate}

\item Instantiate tables and attached fields.

\item Initialize context default values. This has to be done before
the command line is evaluated, otherwise the user settings will be
overwritten.

\item Read command line. Since input/output files are specified on the
command line, it has to be evaluated before the tables are
initialized.

\item Create the algorithm.

\item Initialize the tables (e.g.\ \verb+data.init();+). 

\item Modify algorithm to respect iterations (must be called after
command line iteration).

\item Run algorithm.

\item Close tables.

\end{enumerate}



\chapter{Algorithms}

%DAA = Data Access Algorithms

beide parallel abarbeiten, d.h. 1. von data und 1. von model, dann 2. von data
und 2. von model, etc.
 
\begin{verbatim}
forall(\&data,
  forall(\&model,x()))

verschaltelt abarbeiten:

forall(\&data,
 embed(
  forall(\&model,x())))

nacheinander abarbeiten

forall(\&data,
 seq(
  forall(\&model,x())))
\end{verbatim}

Eigentlich sollte jeder Algorithmus zwei Nachfolge-Algorithmen haben:
einer, der parallel ausgefuehrt wird und einer der sequentiell
ausgefuehrt wird!!! (so wie es jetzt bei allen Iteration und Elapsed
schon ist). Bei einigen Alg. aber nicht so sinnig, etwa
Distanzberechnung: wird 1x gemacht, Ergebnis steht nach step() zur
Verfuegung, seq. und par. Nachfolger sehr aehnlich.


The algorithms in this package do not use buffered tables. This means that all
tables are read from the ASCII multiple times if necessary. However, you can
change that by using {\tt BMasterTable table(...);} instead of {\tt MasterTable
table(...);} and add an {\tt table.submit();} before {\tt table.done(); }

SHARED fields shall not be used in buffered tables.

warum nicht entkopplung von sequentialisierung und algorithmen? (Warum
hat jeder alg einen successor, man koennte die alg doch auch von
aussen verknuepfen, etwa durch seq(a,b) statt bei instantiierung von
a bereits b als successor anzugeben?) Der Vorteil waere, dass bei
mehrfacher verwendung desselben algorithmus dieser mehrfach als
Referenz uebergeben werden koennte (weil ja der Nachfolger
unterschiedlich sein kann ist das jetzt nicht moeglich).

ist mit vorsicht zu geniessen: ein algorithmus \verb+forall(&model)+ mag
mehrfach auftauchen, aber es darf nicht dieselbe Referenz genommen
werden, weil jede Instanz eigene lokale Variablen hat! Um hier
Fehlerquellen zu vermeiden, muss jeder Alg neu instantiiert werden (es
werden ja nur die lokalen Variablen neu angelegt).


grundsaetzlich sollten algorithmen keine lokalen variablen haben
(sondern \verb+abstract_data_types+ dafuer benutzen). Warum? zum beispiel
mean-berechnung: im FCM werden viele means berechnet, iteration ueber
prototypen. wuerde MeanCalc lokale Variablen benutzen, wuerden die
beim wechsel der prototypen durcheinanderkommen (\verb+xembed_loop+)...
man sollte wenigstens pro algorithmus dazuschreiben, ob er lokale Vars
hat, weil das die einsatzmoeglichkeiten beeinflusst


\section{Handling Program Arguments}

To ensure a consistent argument handling, the main program shall not
fetch arguments explicitly. (Note that in some cases, summary
collections may be conceptually equivalent to argument collections.)
Then, the program uses default argument values only, as well as
overridden values such as {\tt args/count=7}. 

Formally, {\tt args<console} does not cause an iteration over the
argument table, therefore no arguments will be read from the command
line but still the default values are used! Only if {\tt
args(*)<console} is used, an automatic algorithm iteration is invoked
and argument records are read from the console. (Consider
identification of {\tt args<console} and {\tt args(*)<console}.)

\chapter{�bersicht}
\input{namespaces}
\section{Class Hierarchy}
This inheritance list is sorted roughly, but not completely, alphabetically\+:\begin{DoxyCompactList}
\item \contentsline{section}{Chars}{\pageref{class_chars}}{}
\item \contentsline{section}{Deleter$<$ T $>$}{\pageref{struct_deleter}}{}
\item \contentsline{section}{less\+\_\+dereference$<$ P\+O\+I\+N\+T\+E\+R $>$}{\pageref{structless__dereference}}{}
\item \contentsline{section}{string\+\_\+less}{\pageref{structstring__less}}{}
\item \contentsline{section}{virtual\+\_\+binary\+\_\+function$<$ Arg1, Arg2, Result $>$}{\pageref{structvirtual__binary__function}}{}
\item \contentsline{section}{virtual\+\_\+unary\+\_\+function$<$ Arg, Result $>$}{\pageref{structvirtual__unary__function}}{}
\end{DoxyCompactList}

\section{Class List}
Here are the classes, structs, unions and interfaces with brief descriptions\+:\begin{DoxyCompactList}
\item\contentsline{section}{\hyperlink{class_chars}{Chars} }{\pageref{class_chars}}{}
\item\contentsline{section}{\hyperlink{struct_deleter}{Deleter$<$ T $>$} }{\pageref{struct_deleter}}{}
\item\contentsline{section}{\hyperlink{structless__dereference}{less\+\_\+dereference$<$ P\+O\+I\+N\+T\+E\+R $>$} }{\pageref{structless__dereference}}{}
\item\contentsline{section}{\hyperlink{structstring__less}{string\+\_\+less} }{\pageref{structstring__less}}{}
\item\contentsline{section}{\hyperlink{structvirtual__binary__function}{virtual\+\_\+binary\+\_\+function$<$ Arg1, Arg2, Result $>$} }{\pageref{structvirtual__binary__function}}{}
\item\contentsline{section}{\hyperlink{structvirtual__unary__function}{virtual\+\_\+unary\+\_\+function$<$ Arg, Result $>$} }{\pageref{structvirtual__unary__function}}{}
\end{DoxyCompactList}

%\section{File List}
Here is a list of all documented files with brief descriptions\+:\begin{DoxyCompactList}
\item\contentsline{section}{/home/liuye/\+Documents/\+Temporal\+Pattern/utility/src/\hyperlink{bool3_8hpp}{bool3.\+hpp} \\*Three-\/valued boolean type }{\pageref{bool3_8hpp}}{}
\item\contentsline{section}{/home/liuye/\+Documents/\+Temporal\+Pattern/utility/src/\hyperlink{_chars_8hpp}{Chars.\+hpp} \\*A simple class to encapsulate char $\ast$ handling }{\pageref{_chars_8hpp}}{}
\item\contentsline{section}{/home/liuye/\+Documents/\+Temporal\+Pattern/utility/src/\hyperlink{defbool_8cpp}{defbool.\+cpp} \\*Implementation of bool-\/char conversions }{\pageref{defbool_8cpp}}{}
\item\contentsline{section}{/home/liuye/\+Documents/\+Temporal\+Pattern/utility/src/\hyperlink{defbool_8hpp}{defbool.\+hpp} \\*Include this file if you use a {\ttfamily bool} data type }{\pageref{defbool_8hpp}}{}
\item\contentsline{section}{/home/liuye/\+Documents/\+Temporal\+Pattern/utility/src/\hyperlink{define_8hpp}{define.\+hpp} \\*Types and constants }{\pageref{define_8hpp}}{}
\item\contentsline{section}{/home/liuye/\+Documents/\+Temporal\+Pattern/utility/src/\hyperlink{stlstream_8cpp}{stlstream.\+cpp} \\*Definition of some constants for templates in \hyperlink{stlstream_8hpp}{stlstream.\+hpp} }{\pageref{stlstream_8cpp}}{}
\item\contentsline{section}{/home/liuye/\+Documents/\+Temporal\+Pattern/utility/src/\hyperlink{stlstream_8hpp}{stlstream.\+hpp} \\*I/\+O routines for S\+T\+L containers }{\pageref{stlstream_8hpp}}{}
\item\contentsline{section}{/home/liuye/\+Documents/\+Temporal\+Pattern/utility/src/\hyperlink{stlutil_8hpp}{stlutil.\+hpp} \\*Implementation of \hyperlink{structstring__less}{string\+\_\+less} }{\pageref{stlutil_8hpp}}{}
\item\contentsline{section}{/home/liuye/\+Documents/\+Temporal\+Pattern/utility/src/\hyperlink{strconv_8cpp}{strconv.\+cpp} \\*Implementation of string conversion routines }{\pageref{strconv_8cpp}}{}
\item\contentsline{section}{/home/liuye/\+Documents/\+Temporal\+Pattern/utility/src/\hyperlink{strconv_8hpp}{strconv.\+hpp} \\*String conversion routines for char$\ast$, string, and \hyperlink{class_chars}{Chars} }{\pageref{strconv_8hpp}}{}
\item\contentsline{section}{/home/liuye/\+Documents/\+Temporal\+Pattern/utility/src/\hyperlink{ttime_8cpp}{ttime.\+cpp} \\*A couple of C functions providing floating point time }{\pageref{ttime_8cpp}}{}
\item\contentsline{section}{/home/liuye/\+Documents/\+Temporal\+Pattern/utility/src/\hyperlink{ttime_8hpp}{ttime.\+hpp} \\*A couple of C functions provide floating point time representation }{\pageref{ttime_8hpp}}{}
\item\contentsline{section}{/home/liuye/\+Documents/\+Temporal\+Pattern/utility/src/{\bfseries ustream-\/d.\+hpp} }{\pageref{ustream-d_8hpp}}{}
\item\contentsline{section}{/home/liuye/\+Documents/\+Temporal\+Pattern/utility/src/{\bfseries ustream-\/s.\+hpp} }{\pageref{ustream-s_8hpp}}{}
\item\contentsline{section}{/home/liuye/\+Documents/\+Temporal\+Pattern/utility/src/\hyperlink{ustream_8cpp}{ustream.\+cpp} \\*Stream i/o }{\pageref{ustream_8cpp}}{}
\item\contentsline{section}{/home/liuye/\+Documents/\+Temporal\+Pattern/utility/src/\hyperlink{ustream_8hpp}{ustream.\+hpp} \\*Stream i/o }{\pageref{ustream_8hpp}}{}
\item\contentsline{section}{/home/liuye/\+Documents/\+Temporal\+Pattern/utility/src/\hyperlink{utiltest_8cpp}{utiltest.\+cpp} \\*Main program to test some functions of the library }{\pageref{utiltest_8cpp}}{}
\end{DoxyCompactList}

\chapter{Klassen-Dokumentation}
\input{class_data_access_library__iobase}
\input{class_data_access_library__writer}
\input{struct_data_access_library__marker_type}
%\chapter{DAL Reference Datei-Dokumentation}
%\input{access_cpp}
%\include{access_hpp}
\printindex

\end{document}
